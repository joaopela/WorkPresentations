\documentclass[8pt]{beamer}
\usepackage[utf8]{inputenc}
\usepackage{xcolor}
\usepackage{colortbl}
\usepackage{epsfig}
\usepackage{cancel}
\usepackage{ulem}
\usepackage{threeparttable} % Joao Pela: 
\usepackage{amsmath}
\usepackage{hyperref}
\usepackage{feynmp}         % For latex produced Feynman Diagrams

% Rule for feynmp diagrams to be considered graphics
\DeclareGraphicsRule{*}{mps}{*}{}

% New compile sequence for feynmp
\makeatletter
\def\endfmffile{%
  \fmfcmd{\p@rcent\space the end.^^J%
          end.^^J%
          endinput;}%
  \if@fmfio
    \immediate\closeout\@outfmf
  \fi
  \ifnum\pdfshellescape=\@ne
    \immediate\write18{mpost \thefmffile}%
  \fi}
\makeatother

\usetheme{Madrid}

\author[João Pela]{J. Pela}
\title[Spin Studies Update]{Spin Studies Update}
%\subtitle{PhD 3 Months Report}
\institute{Imperial College London}
\date{2013-03-01}

% The log drawn in the upper right corner.
\logo{\includegraphics[height=0.115\paperheight]{img/Logo_CMSICL.png}}

\begin{document}
\setlength{\unitlength}{1mm}

\begin{frame}
  \titlepage
\end{frame}

\begin{frame}{Work being developed}

  \begin{block}{Since Last Presentations}
    
    \begin{itemize}
      \item Major code rewrite
      \begin{itemize}
        \item Allow implementation of the CiC based analysis (like fabrice)
        \item Allow use of external configuration files (which allows submitting to batch with different parameters)
      \end{itemize}
      \item Created bin optimization code.
    \end{itemize}

  \end{block}

  \begin{block}{Last week}
 
    \begin{itemize}
      \item Working on implementation of CiC based analysis (being done)
      \item Fixing bug with simultaneous fit. (done)
      \item Moving to Moriond dataset.
    \end{itemize}
 
  \end{block}

\end{frame}

\begin{frame}{Analysis definition: Mass Factorized Based}

  \begin{block}{Analysis Flow}
 
    \begin{itemize}
      \item Start from the categorization of the \textit{Mass Factorized} Analysis taking categories 0 to 3 
      \item Split events on those categories into $cos(\theta^*)$ bins.
      \item Produce dataset over new categories ($4 \times 2$) from data and MC signal samples
      \item Create and fit Signal Model ($3 \times$ Gaussian) to MC signal SM 
      \item Fit same model to Alternative Model (Spin 2)
      \item Fit background Model to data sideband and extrapolate yield at signal region
      \item Compute separation from previous values (to be done)
      \item Fix all parameters on all the categories of Signal Model
      \item Fit Signal Model (floating signal strength $\mu$) + Background Model to data.
    \end{itemize}
 
  \end{block}

    \begin{block}{Details}
 
    \begin{itemize}
      \item Signal area $2\%$ around 125 $GeV$.
    \end{itemize}
 
    \end{block}

    \begin{block}{To be done:}
 
    \begin{itemize}
      \item Normalization of Alternative Model (Spin 2) to SM Model total number of events.
      \item Pass event Yields to separation code 
    \end{itemize}
 
    \end{block}

\end{frame}

\begin{frame}{Analysis definition: Cut based}

  \begin{block}{Analysis Flow}
 
    \begin{itemize}
      \item Start from the categorization of the \textit{Cut Based} Analysis taking categories 0 to 3 
      \item Split events on those categories into 5 $cos(\theta^*)$ bins (0.2 spacing).
      \item Produce dataset over new categories ($4 \times 5$) from data and MC signal samples
      \item Create and fit Signal Model ($3 \times$ Gaussian) to MC signal SM 
      \item Create efficiency function to flat out $cos(\theta^*)$ for SM 
      \item Fit Signal Model to Alternative Sample (Spin 2) and apply efficiency correction (to be done)
      \item Fit background Model to data sideband and extrapolate yield at signal region
      \item Compute separation from previous values (to be done)
      \item Fix all parameters on all the categories of Signal Model (to be done)
      \item Fit Signal Model (floating signal strength $\mu$) + Background Model to data. (to be done)
    \end{itemize}
 
  \end{block}

    \begin{block}{Details}
 
    \begin{itemize}
      \item Signal area $2\%$ around 125 $GeV$.
    \end{itemize}
 
    \end{block}

    \begin{block}{To be done:}
 
    \begin{itemize}
      \item Normalization of Alternative Model (Spin 2) to SM Model total number of events.
      \item Pass event Yields to separation code 
    \end{itemize}
 
    \end{block}

\end{frame}

\begin{frame}{MassFac: Signal Model Fit SM}
  
  
  
\end{frame}

\begin{frame}{MassFac: Background Model Fit SM}
  
\end{frame}

\begin{frame}{MassFac: Data obtained $\mu$}
  
\end{frame}

\begin{frame}{Summary:}
 
  \begin{block}{Last week}

  \begin{itemize}
    \item Stuck for a long time solving simultaneous fit problem, fixed late yesterday.
    \item Moved to Moriond ntuples.
    \item Finishing implementation of the CiC analysis.
  \end{itemize}

  \end{block}

  \begin{block}{Next:}

    \begin{itemize}
      \item Finish the CiC Analysis implementation
      \item Pass values from both analysis to Matt's code for separation
      \item Run boundary optimization code.
      \item Redo analysis with new boundaries
    \end{itemize}

  \end{block}
  
\end{frame}

\end{document}