\documentclass[8pt]{beamer}
\usepackage[utf8]{inputenc}
\usepackage{xcolor}
\usepackage{colortbl}
\usepackage{epsfig}
\usepackage{cancel}
\usepackage{ulem}
\usepackage{threeparttable} % Joao Pela: 
\usepackage{amsmath}
\usepackage{hyperref}
\usepackage{feynmp}         % For latex produced Feynman Diagrams

% Rule for feynmp diagrams to be considered graphics
\DeclareGraphicsRule{*}{mps}{*}{}

% New compile sequence for feynmp
\makeatletter
\def\endfmffile{%
  \fmfcmd{\p@rcent\space the end.^^J%
          end.^^J%
          endinput;}%
  \if@fmfio
    \immediate\closeout\@outfmf
  \fi
  \ifnum\pdfshellescape=\@ne
    \immediate\write18{mpost \thefmffile}%
  \fi}
\makeatother

\usetheme{Madrid}

\author[João Pela]{J. Pela}
\title[Spin Studies Update]{Spin Studies Update}
%\subtitle{PhD 3 Months Report}
\institute{Imperial College London}
\date{2012-11-30}

% The log drawn in the upper right corner.
\logo{\includegraphics[height=0.115\paperheight]{img/Logo_CMSICL.png}}

\begin{document}
\setlength{\unitlength}{1mm}

\begin{frame}
  \titlepage
\end{frame}

\begin{frame}{Since Last Presentations}

  \begin{block}{On the Spin Analysis}
    
    \begin{itemize}
      \item Moved to HCP dataset
      \begin{itemize}
        \item More Data...
        \item Same graviton sample...
      \end{itemize}
      \item Looking into variables to explain difference on Yield total/relative between different signals. 
      \item Create a CVS area for current code (now shared between me and Matt).
      \item Implemented several fixes and improvements to analysis code.
    \end{itemize}

  \end{block}

  \begin{block}{Other activities}
  
    \begin{itemize}
      \item Finished my central shifts for 2012
      \item Created/Tested/Deployed new CMSSW package tag for L1 DQM Offline.
    \end{itemize}

  \end{block}

\end{frame}

\begin{frame}{Looking into relevant variables}

  \begin{block}{Why?}
   
    \begin{itemize}
      \item In the current analysis we start from the assumption the SM Higgs and Graviton cross section is the same.
      \item After minimal diphoton BDT cuts ($score > -0.05$) the Graviton yield is approximately half the SM Higgs.
      \item It would be important to determine which are the variables that make more graviton events get rejected or
      moved to lower BDT categories.
    \end{itemize}
   
  \end{block}

  \begin{block}{What?}
   
    Looked at:
    \begin{itemize}
      \item Contribution to mass peak of events between barrel and endcap
      \item $p_T$ and $\eta$ of lead/sublead photon
      \item $p_T$ and $\eta$ diphoton
      \item Minimum and Maximum photon $\eta$
    \end{itemize}
   
  \end{block}
  
\end{frame}

\begin{frame}{Changes ICHEP to HCP analysis}
 
   \begin{block}{Differences Main Analysis}
   
    \begin{itemize}
      \item All the changes/new features of the main HCP analysis.
      \item Including the new calibration and VBF MVA.
    \end{itemize}
   
  \end{block}

     \begin{block}{Differences Spin Analysis}
   
    \begin{itemize}
      \item New optimized $cos(\theta *)$ bins. Ordered by BDT category:
      \begin{itemize}
        \item ICHEP: 0.7, 0.3, 0.4, 0.5
        \item HCP: 0.5, 0.4, 0.4, 0.5
      \end{itemize}
    \end{itemize}
   
  \end{block}
  
\end{frame}

\begin{frame}{Lead photon $p_T$}

  \begin{block}
   
     \begin{columns}
     
      \centering
     
      \column[t]{3.0cm}
      \includegraphics[width=1.00\textwidth]{plots/leadPhotonPt_bdt0_cTh0.pdf} \\
      \includegraphics[width=1.00\textwidth]{plots/leadPhotonPt_bdt0_cTh1.pdf} 
      
      \column[t]{3.0cm}
      \includegraphics[width=1.00\textwidth]{plots/leadPhotonPt_bdt1_cTh0.pdf} \\
      \includegraphics[width=1.00\textwidth]{plots/leadPhotonPt_bdt1_cTh1.pdf}
      
      \column[t]{3.0cm}
      \includegraphics[width=1.00\textwidth]{plots/leadPhotonPt_bdt2_cTh0.pdf} \\
      \includegraphics[width=1.00\textwidth]{plots/leadPhotonPt_bdt2_cTh1.pdf}
      
      \column[t]{3.0cm}
      \includegraphics[width=1.00\textwidth]{plots/leadPhotonPt_bdt3_cTh0.pdf} \\
      \includegraphics[width=1.00\textwidth]{plots/leadPhotonPt_bdt3_cTh1.pdf}
      
    \end{columns}
  
  \end{block}

  \begin{itemize}
    \item In all categories SM Higgs and graviton are compatible within statistical errors
  \end{itemize}

\end{frame}

\begin{frame}{Lead photon $\eta$}

  \begin{block}
   
     \begin{columns}
     
      \centering
     
      \column[t]{3.0cm}
      \includegraphics[width=1.00\textwidth]{plots/leadPhotonEta_bdt0_cTh0.pdf} \\
      \includegraphics[width=1.00\textwidth]{plots/leadPhotonEta_bdt0_cTh1.pdf} 
      
      \column[t]{3.0cm}
      \includegraphics[width=1.00\textwidth]{plots/leadPhotonEta_bdt1_cTh0.pdf} \\
      \includegraphics[width=1.00\textwidth]{plots/leadPhotonEta_bdt1_cTh1.pdf}
      
      \column[t]{3.0cm}
      \includegraphics[width=1.00\textwidth]{plots/leadPhotonEta_bdt2_cTh0.pdf} \\
      \includegraphics[width=1.00\textwidth]{plots/leadPhotonEta_bdt2_cTh1.pdf}
      
      \column[t]{3.0cm}
      \includegraphics[width=1.00\textwidth]{plots/leadPhotonEta_bdt3_cTh0.pdf} \\
      \includegraphics[width=1.00\textwidth]{plots/leadPhotonEta_bdt3_cTh1.pdf}
      
    \end{columns}
  
  \end{block}

    \begin{itemize}
      \item In all categories SM Higgs and graviton are compatible within statistical errors
      \item But High $cos(\theta *)$ plots show a different shape, peaking at $|eta| \sim 1$
    \end{itemize}
  
\end{frame}

\begin{frame}{Sublead photon $p_T$}

  \begin{block}
   
     \begin{columns}
     
      \centering
     
      \column[t]{3.0cm}
      \includegraphics[width=1.00\textwidth]{plots/subleadPhotonPt_bdt0_cTh0.pdf} \\
      \includegraphics[width=1.00\textwidth]{plots/subleadPhotonPt_bdt0_cTh1.pdf} 
      
      \column[t]{3.0cm}
      \includegraphics[width=1.00\textwidth]{plots/subleadPhotonPt_bdt1_cTh0.pdf} \\
      \includegraphics[width=1.00\textwidth]{plots/subleadPhotonPt_bdt1_cTh1.pdf}
      
      \column[t]{3.0cm}
      \includegraphics[width=1.00\textwidth]{plots/subleadPhotonPt_bdt2_cTh0.pdf} \\
      \includegraphics[width=1.00\textwidth]{plots/subleadPhotonPt_bdt2_cTh1.pdf}
      
      \column[t]{3.0cm}
      \includegraphics[width=1.00\textwidth]{plots/subleadPhotonPt_bdt3_cTh0.pdf} \\
      \includegraphics[width=1.00\textwidth]{plots/subleadPhotonPt_bdt3_cTh1.pdf}
      
    \end{columns}
  
  \end{block}

  \begin{itemize}
    \item In all categories SM Higgs and graviton are compatible within statistical errors
  \end{itemize}
  
\end{frame}

\begin{frame}{Sublead photon $\eta$}

  \begin{block}
   
     \begin{columns}
     
      \centering
     
      \column[t]{3.0cm}
      \includegraphics[width=1.00\textwidth]{plots/subleadPhotonEta_bdt0_cTh0.pdf} \\
      \includegraphics[width=1.00\textwidth]{plots/subleadPhotonEta_bdt0_cTh1.pdf} 
      
      \column[t]{3.0cm}
      \includegraphics[width=1.00\textwidth]{plots/subleadPhotonEta_bdt1_cTh0.pdf} \\
      \includegraphics[width=1.00\textwidth]{plots/subleadPhotonEta_bdt1_cTh1.pdf}
      
      \column[t]{3.0cm}
      \includegraphics[width=1.00\textwidth]{plots/subleadPhotonEta_bdt2_cTh0.pdf} \\
      \includegraphics[width=1.00\textwidth]{plots/subleadPhotonEta_bdt2_cTh1.pdf}
      
      \column[t]{3.0cm}
      \includegraphics[width=1.00\textwidth]{plots/subleadPhotonEta_bdt3_cTh0.pdf} \\
      \includegraphics[width=1.00\textwidth]{plots/subleadPhotonEta_bdt3_cTh1.pdf}
      
    \end{columns}
  
  \end{block}

  \begin{itemize}
    \item In all categories SM Higgs and graviton are compatible within statistical errors
    \item But High $cos(\theta *)$ plots show a different shape, peaking at $|eta| \sim 1$
  \end{itemize}
  
\end{frame}

\begin{frame}{Diphoton $p_T$}

  \begin{block}
   
     \begin{columns}
     
      \centering
     
      \column[t]{3.0cm}
      \includegraphics[width=1.00\textwidth]{plots/diphotonPt_bdt0_cTh0.pdf} \\
      \includegraphics[width=1.00\textwidth]{plots/diphotonPt_bdt0_cTh1.pdf} 
      
      \column[t]{3.0cm}
      \includegraphics[width=1.00\textwidth]{plots/diphotonPt_bdt1_cTh0.pdf} \\
      \includegraphics[width=1.00\textwidth]{plots/diphotonPt_bdt1_cTh1.pdf}
      
      \column[t]{3.0cm}
      \includegraphics[width=1.00\textwidth]{plots/diphotonPt_bdt2_cTh0.pdf} \\
      \includegraphics[width=1.00\textwidth]{plots/diphotonPt_bdt2_cTh1.pdf}
      
      \column[t]{3.0cm}
      \includegraphics[width=1.00\textwidth]{plots/diphotonPt_bdt3_cTh0.pdf} \\
      \includegraphics[width=1.00\textwidth]{plots/diphotonPt_bdt3_cTh1.pdf}
      
    \end{columns}
  
  \end{block}

  \begin{itemize}
    \item In all categories SM Higgs and graviton are compatible within statistical errors
    \item But High $cos(\theta *)$ (bdt=1,2) plots show a different shape, peaking at more sharply and decaying faster for Graviton
  \end{itemize}
  
\end{frame}

\begin{frame}{Diphoton $\eta$}

  \begin{block}
   
     \begin{columns}
     
      \centering
     
      \column[t]{3.0cm}
      \includegraphics[width=1.00\textwidth]{plots/diphotonEta_bdt0_cTh0.pdf} \\
      \includegraphics[width=1.00\textwidth]{plots/diphotonEta_bdt0_cTh1.pdf} 
      
      \column[t]{3.0cm}
      \includegraphics[width=1.00\textwidth]{plots/diphotonEta_bdt1_cTh0.pdf} \\
      \includegraphics[width=1.00\textwidth]{plots/diphotonEta_bdt1_cTh1.pdf}
      
      \column[t]{3.0cm}
      \includegraphics[width=1.00\textwidth]{plots/diphotonEta_bdt2_cTh0.pdf} \\
      \includegraphics[width=1.00\textwidth]{plots/diphotonEta_bdt2_cTh1.pdf}
      
      \column[t]{3.0cm}
      \includegraphics[width=1.00\textwidth]{plots/diphotonEta_bdt3_cTh0.pdf} \\
      \includegraphics[width=1.00\textwidth]{plots/diphotonEta_bdt3_cTh1.pdf}
      
    \end{columns}
  
  \end{block}

    \begin{itemize}
    \item In all categories SM Higgs and graviton are compatible within statistical errors
    \item Several categories show significant shape differences
  \end{itemize}
  
\end{frame}

\begin{frame}{Minimum photon $\eta$}

  \begin{block}
   
     \begin{columns}
     
      \centering
     
      \column[t]{3.0cm}
      \includegraphics[width=1.00\textwidth]{plots/diphotonMinEta_bdt0_cTh0.pdf} \\
      \includegraphics[width=1.00\textwidth]{plots/diphotonMinEta_bdt0_cTh1.pdf} 
      
      \column[t]{3.0cm}
      \includegraphics[width=1.00\textwidth]{plots/diphotonMinEta_bdt1_cTh0.pdf} \\
      \includegraphics[width=1.00\textwidth]{plots/diphotonMinEta_bdt1_cTh1.pdf}
      
      \column[t]{3.0cm}
      \includegraphics[width=1.00\textwidth]{plots/diphotonMinEta_bdt2_cTh0.pdf} \\
      \includegraphics[width=1.00\textwidth]{plots/diphotonMinEta_bdt2_cTh1.pdf}
      
      \column[t]{3.0cm}
      \includegraphics[width=1.00\textwidth]{plots/diphotonMinEta_bdt3_cTh0.pdf} \\
      \includegraphics[width=1.00\textwidth]{plots/diphotonMinEta_bdt3_cTh1.pdf}
      
    \end{columns}
  
  \end{block}

  \begin{itemize}
    \item In all categories SM Higgs and graviton are compatible within statistical errors
    \item High $cos(\theta *)$ seem bins to show that on the photons is always more central
  \end{itemize}
  
\end{frame}


\begin{frame}{Maximum photon $\eta$}

  \begin{block}
   
     \begin{columns}
     
      \centering
     
      \column[t]{3.0cm}
      \includegraphics[width=1.00\textwidth]{plots/diphotonMaxEta_bdt0_cTh0.pdf} \\
      \includegraphics[width=1.00\textwidth]{plots/diphotonMaxEta_bdt0_cTh1.pdf} 
      
      \column[t]{3.0cm}
      \includegraphics[width=1.00\textwidth]{plots/diphotonMaxEta_bdt1_cTh0.pdf} \\
      \includegraphics[width=1.00\textwidth]{plots/diphotonMaxEta_bdt1_cTh1.pdf}
      
      \column[t]{3.0cm}
      \includegraphics[width=1.00\textwidth]{plots/diphotonMaxEta_bdt2_cTh0.pdf} \\
      \includegraphics[width=1.00\textwidth]{plots/diphotonMaxEta_bdt2_cTh1.pdf}
      
      \column[t]{3.0cm}
      \includegraphics[width=1.00\textwidth]{plots/diphotonMaxEta_bdt3_cTh0.pdf} \\
      \includegraphics[width=1.00\textwidth]{plots/diphotonMaxEta_bdt3_cTh1.pdf}
      
    \end{columns}
  
  \end{block}

    \begin{itemize}
    \item In all categories SM Higgs and graviton are compatible within statistical errors
    \item High $cos(\theta *)$ bins as expected have higher values
  \end{itemize}
  
\end{frame}

\begin{frame}{Conclusions}
 
  \begin{block}{Conclusions}

  \begin{itemize}
    \item Significant shape differences where found in $p_T$ and $\eta$ distributions but suffering from low sample statistics.
  \end{itemize}

  \end{block}

  \begin{block}{Next}

    \begin{itemize}
      \item Redo all plots with HCP dataset
      \item Investigate more variables
      \item Look into other possible event selection methodologies (example: cut based)
    \end{itemize}

  \end{block}
  
\end{frame}

\end{document}