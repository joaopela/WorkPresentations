\documentclass[8pt]{beamer}
\usepackage[utf8]{inputenc}
\usepackage{xcolor}
\usepackage{colortbl}
\usepackage{epsfig}
\usepackage{cancel}
\usepackage{ulem}
\usepackage{threeparttable} % Joao Pela: 
\usepackage{amsmath}
\usepackage{hyperref}
\usepackage{feynmp}         % For latex produced Feynman Diagrams
\usepackage{graphicx}

% Rule for feynmp diagrams to be considered graphics
\DeclareGraphicsRule{*}{mps}{*}{}

% New compile sequence for feynmp
\makeatletter
\def\endfmffile{%
  \fmfcmd{\p@rcent\space the end.^^J%
          end.^^J%
          endinput;}%
  \if@fmfio
    \immediate\closeout\@outfmf
  \fi
  \ifnum\pdfshellescape=\@ne
    \immediate\write18{mpost \thefmffile}%
  \fi}
\makeatother

\usetheme{Madrid}

\author[João Pela]{J. Pela}
\title[Spin Studies Update]{Spin Studies Update}
%\subtitle{PhD 3 Months Report}
\institute{Imperial College London}
\date{2013-03-08}

% The log drawn in the upper right corner.
\logo{\includegraphics[height=0.115\paperheight]{img/Logo_CMSICL.png}}

\begin{document}
\setlength{\unitlength}{1mm}

\begin{frame}
  \titlepage
\end{frame}

% \begin{frame}{Work being developed}
% 
%   \begin{block}{Since Last Presentations}
%     
%     \begin{itemize}
%       \item 
%     \end{itemize}
% 
%   \end{block}
% 
% \end{frame}
% 
% \begin{frame}{Analysis definition: Mass Factorized Based}
% 
%   \begin{block}{Analysis Flow}
%  
%     \begin{itemize}
%       \item Start from the categorization of the \textit{Mass Factorized} Analysis taking categories 0 to 3 
%       \item Split events on those categories into $cos(\theta^*)$ bins.
%       \item Produce dataset over new categories ($4 \times 2$) from data and MC signal samples
%       \item Create and fit Signal Model ($3 \times$ Gaussian) to MC signal SM 
%       \item Fit same model to Alternative Model (Spin 2)
%       \item Fit background Model to data sideband and extrapolate yield at signal region
%       \item Compute separation from previous values (to be done)
%       \item Fix all parameters on all the categories of Signal Model
%       \item Fit Signal Model (floating signal strength $\mu$) + Background Model to data.
%     \end{itemize}
%  
%   \end{block}
% 
%     \begin{block}{Details}
%  
%     \begin{itemize}
%       \item Signal area $2\%$ around 125 $GeV$.
%     \end{itemize}
%  
%     \end{block}
% 
%     \begin{block}{To be done:}
%  
%     \begin{itemize}
%       \item Normalization of Alternative Model (Spin 2) to SM Model total number of events.
%       \item Pass event Yields to separation code 
%     \end{itemize}
%  
%     \end{block}
% 
% \end{frame}
% 
% \begin{frame}{Analysis definition: Cut based}
% 
%   \begin{block}{Analysis Flow}
%  
%     \begin{itemize}
%       \item Start from the categorization of the \textit{Cut Based} Analysis taking categories 0 to 3 
%       \item Split events on those categories into 5 $cos(\theta^*)$ bins (0.2 spacing).
%       \item Produce dataset over new categories ($4 \times 5$) from data and MC signal samples
%       \item Create and fit Signal Model ($3 \times$ Gaussian) to MC signal SM 
%       \item Create efficiency function to flat out $cos(\theta^*)$ for SM 
%       \item Fit Signal Model to Alternative Sample (Spin 2) and apply efficiency correction (to be done)
%       \item Fit background Model to data sideband and extrapolate yield at signal region
%       \item Compute separation from previous values (to be done)
%       \item Fix all parameters on all the categories of Signal Model (to be done)
%       \item Fit Signal Model (floating signal strength $\mu$) + Background Model to data. (to be done)
%     \end{itemize}
%  
%   \end{block}
% 
%     \begin{block}{Details}
%  
%     \begin{itemize}
%       \item Signal area $2\%$ around 125 $GeV$.
%     \end{itemize}
%  
%     \end{block}
% 
%     \begin{block}{To be done:}
%  
%     \begin{itemize}
%       \item Normalization of Alternative Model (Spin 2) to SM Model total number of events.
%       \item Pass event Yields to separation code 
%     \end{itemize}
%  
%     \end{block}
% 
% \end{frame}

\begin{frame}{Normalization of MC Models}

  For consistency the Model B (Spin 2 Model) is normalized to the total number of events over all categories after diphoton BDT 
  cut. 
  \begin{block}{Absolute Values}

    \resizebox{\linewidth}{!}{
    \begin{table}
% \centering
\begin{tabular}{|c||c|c|c|c|c|c|c|c|c|}
\hline
\multicolumn{10}{|c|}{Model Normalization} \\ 
\hline
 & $bin^{BDT 0}_{CTh 0}$ & $bin^{BDT 0}_{CTh 1}$ & $bin^{BDT 1}_{CTh 0}$ & $bin^{BDT 1}_{CTh 1}$ & $bin^{BDT 2}_{CTh 0}$ & $bin^{BDT 2}_{CTh 1}$ & $bin^{BDT 3}_{CTh 0}$ & $bin^{BDT 3}_{CTh 1}$ & Total \\ 
\hline
\hline
Model A &$   8.924$ & $   3.199$ & $  25.869$ & $   9.856$ & $ 103.603$ & $  43.781$ & $  95.076$ & $  60.979$ & $ 351.288$ \\
Model B (before) &$   4.802$ & $   3.003$ & $  12.602$ & $   8.761$ & $  50.401$ & $  39.061$ & $  52.351$ & $  67.679$ & $ 238.660$ \\
Model B (after) &$   7.068$ & $   4.420$ & $  18.549$ & $  12.896$ & $  74.186$ & $  57.494$ & $  77.056$ & $  99.618$ & $ 351.288$ \\
\hline
\end{tabular}
\end{table}

    }
    
  \end{block}

  \begin{block}{Relative}

    \resizebox{\linewidth}{!}{
    \begin{table}
\centering
\begin{tabular}{|c||c|c|c|c|c|c|c|c|c|}
\hline
\multicolumn{10}{|c|}{Model Normalization (relative)} \\ 
\hline
 & $bin^{BDT 0}_{CTh 0}$ & $bin^{BDT 0}_{CTh 1}$ & $bin^{BDT 1}_{CTh 0}$ & $bin^{BDT 1}_{CTh 1}$ & $bin^{BDT 2}_{CTh 0}$ & $bin^{BDT 2}_{CTh 1}$ & $bin^{BDT 3}_{CTh 0}$ & $bin^{BDT 3}_{CTh 1}$ & Total \\ 
\hline
\hline
Model A &$   0.025$ & $   0.009$ & $   0.074$ & $   0.028$ & $   0.295$ & $   0.125$ & $   0.271$ & $   0.174$ & $   1.000$ \\
Model B (before) &$   0.020$ & $   0.013$ & $   0.053$ & $   0.037$ & $   0.211$ & $   0.164$ & $   0.219$ & $   0.284$ & $   1.000$ \\
Model B (after) &$   0.020$ & $   0.013$ & $   0.053$ & $   0.037$ & $   0.211$ & $   0.164$ & $   0.219$ & $   0.284$ & $   1.000$ \\
\hline
\end{tabular}
\end{table}

    }    

  \end{block}
  
\end{frame}

\begin{frame}{Conclusions}
 
 
\end{frame}

\begin{frame}{Backup Slides}[noframenumbering]
 
 \begin{huge}Backup Slides\end{huge}
 
\end{frame}

\end{document}