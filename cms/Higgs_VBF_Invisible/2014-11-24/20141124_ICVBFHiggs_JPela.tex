\documentclass[8pt]{beamer}
\usepackage[utf8]{inputenc}
\usepackage{ulem}
\usepackage{xcolor}
\usepackage{colortbl}
\usepackage{epsfig}
% \usepackage{cancel}
\usepackage{ulem}
% \usepackage{threeparttable} % Joao Pela: 
\usepackage{amsmath}
\usepackage{hyperref}
% \usepackage{appendixnumberbeamer}
% \usepackage{pdfpages}

\usetheme{Madrid}

\author[J. Pela]{J. Pela}
\title{VBF Higgs to Invisible cross check analysis}
\institute[ICL]{Imperial College London}
\date{2014-11-24}

% The log drawn in the upper right corner.
\logo{\includegraphics[height=0.115\paperheight]{img/Logo_CMSICL.png}}

\begin{document}
\setlength{\unitlength}{1mm}

% ###################################################
\begin{frame}
  \titlepage
\end{frame}

% ###################################################
\begin{frame}{Introduction}

 were produced by two different and independent code frameworks. This a

\begin{block}{2012 prompt results and publication}
 
 \begin{itemize}
  \item Produced by two different and independent code frameworks
  \item Allowed debugging via synchronization
  \item Extra confidence on the final results
 \end{itemize}

\end{block}

\begin{block}{2012 parked results and publication}
 
  Due to lack of man power and time only one full framework. Later decided some level cross check would be beneficial for the analysis. What is being done:
 
 \begin{itemize}
  \item Cross check analysis starts from the same ntuples produced by the main analysis.
  \begin{itemize}
    \item All the relevant datasets with not filtering (except golden JSON), just data reduction and parking.
    \item Production of ntuples uses same framework on other validated analysis: SM and MSSM Higgs $\tau\bar{\tau}$, the Higgs to $\tau\bar{\tau}b\bar{b}$ and also the prompt invisible result.
  \end{itemize}
  
 \end{itemize}
 
\end{block}

\end{frame}

% ###################################################
\begin{frame}{Software}

I have created a git software repository for analysis independent software and tools about a year ago this will be the base for the analysis code:
\begin{block}{Base}
https://github.com/joaopela/HEPFW
\end{block}

This software package was already forked and was the base for the recent Run II trigger studies and now is being used for the cross check analysis.
\begin{block}{Fork}
https://github.com/ICHiggsInv/HEPFW
\end{block}

The target is to have a framework capable of replicating all the relevant pltos and numbers of the main analysis.

\end{frame}

% ###################################################
\begin{frame}{Current status}
 
Framework development was inspired on CMSSW and ROOT structures and features. What is the status:
 
\begin{block}{Capabilities}
 
  \begin{itemize}
   \item Access IC Dataformats
   \item Run over all datasets automatically
   \item Multiple sequence capability
   \item EDM filter filtering
   \item Event list filtering
   \item Collection cuts based filtering
   \item Collection overlap filtering
  \end{itemize}

\end{block}
 
\begin{block}{Analysis specific}

  \begin{itemize}
   \item Vertex selection
   \item Electron selection (and veto)
   \item Muon selection (and veto)
   \item MET cuts
   \item Jet cuts (being finished)
   \item Multiple object cuts ($Min(\Delta\phi(MET,jets))$)
  \end{itemize}

\end{block}
 
\end{frame}

% ###################################################
\begin{frame}{Problems Found: L1T ETM MET cut value}
 

\end{frame}

% ###################################################
\begin{frame}{Problems Found: Primary vertex conditions}
 
\end{frame}

% ###################################################
\begin{frame}{Problems Found: PFJet ID AN-to-Code disagreement}
 
\end{frame}

% ###################################################
\begin{frame}{Documentation issue: Electron additional cuts}
 
\end{frame}

% ###################################################
\begin{frame}{First results - event quality filters}

We apply a set of event quality filters to exclude events with clear problems. This filters cover several issue like bad experimental conditions (like too much beam halo), detector noise and problems with reconstruction. The inclusion of the filter is recommended for analysis using MET.

The following table has the percentage of rejected events by each filter individually before any other cuts are applied to data. At the end the total percentage event rejection (using an or of all filters) is provided. 

\begin{block}{Topics}

\resizebox{\linewidth}{!}{
\begin{tabular}{|l||c|c|c|c||c|c|c|}
\hline
Filter & Prompt A & Prompt B & Prompt C & Prompt D & Parked B & Parked C & Parked D \\
\hline \hline
HBHENoiseFilter & 22.900905 & 22.527028 & 24.197981 & 21.830762 & 0.190670 & 0.187739 & 0.170753 \\
EcalDeadCellTriggerPrimitiveFilter & 0.375381 & 2.106058 & 2.122628 & 0.838167 & 0.009300 & 0.010206 & 0.012526 \\
eeBadScFilter & 0.007852 & 0.002690 & 0.000087 & 0.001883 & 0.000001 & 0.000000 & 0.000009 \\
trackingFailureFilter & 3.073876 & 1.147820 & 1.638249 & 1.723157 & 0.000328 & 0.007464 & 0.000290 \\
manystripclus53X & 0.001829 & 0.004350 & 0.007730 & 0.005510 & 0.001319 & 0.002335 & 0.001327 \\
toomanystripclus53X & 0.000484 & 0.001765 & 0.003141 & 0.001732 & 0.001149 & 0.002006 & 0.001173 \\
logErrorTooManyClusters & 0.000027 & 0.000190 & 0.000379 & 0.000102 & 0.000009 & 0.000021 & 0.000016 \\
CSCTightHaloFilter & 10.263068 & 7.792489 & 6.133730 & 8.346904 & 0.398497 & 0.402936 & 0.508025 \\
\hline \hline
Total & 28.501208 & 28.317395 & 28.978159 & 26.047078 & 0.598417 & 0.601999 & 0.689380 \\
\hline
\end{tabular}}

\end{block}

\begin{block}{Comparison with main analysis:}
 This values can be found on the first part of table 3 on the AN2014\_243\_v4. While total filter efficiency matched exactly the values presented on the note, there are small
 discrepancies on individual filters with very low event exclusion. This discrepancies may be due to rounding problems or double vs float conversions where precision is lot.
\end{block}

\end{frame}

% ###################################################
\begin{frame}{First results - ECAL and HCAL laser  filters}

There are events that should be not be considered due to have been recorded while the ECAL and/or HCAL laser calibration sequence was happening. The identification of this events is provided through a file list which is used by the code framework to remove those events. The following table shows the individual and total percentage of events vetoed out of the the ones already passing the event quality filters of the previous section.

\begin{block}{Topics}

\resizebox{\linewidth}{!}{
\begin{tabular}{|l||c|c|c|c||c|c|c|}
\hline
Filter & Prompt A & Prompt B & Prompt C & Prompt D & Parked B & Parked C & Parked D \\
\hline \hline
ECAL Laser Filter & 0.928521 & 1.195528 & 0.000000 & 0.000000 & 0.008659 & 0.000000 & 0.000000 \\
HCAL Laser Filter & 0.007258 & 0.000027 & 0.004963 & 0.000000 & 0.000000 & 0.000270 & 0.000000 \\
\hline \hline
ECAL+HCAL Laser Filter & 0.935704 & 1.195556 & 0.004963 & 0.000000 & 0.008659 & 0.000270 & 0.000000 \\
\hline
\end{tabular}}

\end{block}

\begin{block}{Comparison with main analysis:}
The "ECAL+HCAL Laser Filter" line of values can be compared with the one on  table 3 on the AN2014\_243\_v4. Values match exactly the ones of the main analysis.
\end{block}

\end{frame}

% ###################################################
\begin{frame}{Summary and next steps}
 
\begin{block}{Summary:}
 
\begin{itemize}
  \item Now able to replicate some number of the analys note.
  \item Some bugs already found (minimal influce in main analysis):
  \begin{itemize}
    \item Good vertex requirement not being requested on main analysis code (effect unknow but should be small)
    \item L1T met cut was $>40$ and not $>=40$
  \end{itemize}
  \item Analysis contribution
  \begin{itemize}
    \item Checking all definitions against POG recommendation
  \end{itemize}
  \end{itemize}

\end{block}

\begin{block}{Next Steps:}
 
\begin{itemize}
  \item Replicate pre-selection results.
\end{itemize}
 
\end{block}

\begin{center}
https://twiki.cern.ch/twiki/bin/viewauth/CMS/VBFHInvParkedDataCrossCheck
\end{center}

\end{frame}

\end{document}
