\documentclass[8pt]{beamer}
\usepackage[utf8]{inputenc}
\usepackage{xcolor}
\usepackage{colortbl}
\usepackage{epsfig}
% \usepackage{cancel}
\usepackage{ulem}
% \usepackage{threeparttable} % Joao Pela: 
\usepackage{amsmath}
\usepackage{hyperref}
% \usepackage{appendixnumberbeamer}
\usepackage{feynmp}         % For latex produced Feynman Diagrams

%Rule for feynmp diagrams to be considered graphics
\DeclareGraphicsRule{*}{mps}{*}{}

% New compile sequence for feynmp
\makeatletter
\def\endfmffile{%
  \fmfcmd{\p@rcent\space the end.^^J%
          end.^^J%
          endinput;}%
  \if@fmfio
    \immediate\closeout\@outfmf
  \fi
  \ifnum\pdfshellescape=\@ne
    \immediate\write18{mpost \thefmffile}%
  \fi}
\makeatother

\usetheme{Madrid}

\author[J. Pela]{João Pela}
\title{Update on the VBF Invisible cross check analysis}
\institute[ICL]{Imperial College London}
\date{2014-12-04}

% The log drawn in the upper right corner.
\logo{\includegraphics[height=0.115\paperheight]{img/Logo_CMSICL.png}}

\begin{document}
\setlength{\unitlength}{1mm}

% ###################################################
\begin{frame}
  \titlepage
\end{frame}

% ###################################################
\begin{frame}{Today's presentation}
 
\begin{block}{Topics}
 
\begin{itemize}
  \item Cross check framework updates
  \item Updated pre-selection yiels
  \item Electron+MET region yields 
\end{itemize}
 
\end{block}

\end{frame}

% ###################################################
\begin{frame}{Cross check framework updates}
 
\begin{block}{Topics}
 
\begin{itemize}
  \item Core event processing code review and simplified
  \item Multiple sequences can be defined and executed in a single job, no module execution repetition.
  \item Using JSON language for configuration files (same role as python in CMSSW)
  \begin{itemize}
    \item Each file can load other configuration files (allows splitting of cfg files)
    \item Per modules/sequence parameter (like CMSSW)
    \item Allows modules sequence definition outside of code (no recompilation)
    \item No hard-coding of an parameter of sequence.
  \end{itemize}
  \item Some limited C++ refection capabilities: Framework detects new (user defined) modules and the become automatically usable via configuration file.  
  \item New modules to produce modified MET (necessary to produce MET-no-Muons and later MET scaling)
\end{itemize}

\end{block}

This new features will allow further implementation of all regions and systematic studies in a faster/modular way. Commissioning was made by reproducing exactly the yields for data at the pre-selection level. 

\end{frame}

% ###################################################
\begin{frame}{Typo found}

While converting framework to the new non-hard-coded structure I review all cuts already implemented and found a typo on the electron veto (specifically on the value of the relative combined isolation) which I corrected.
 
\begin{block}{Pre-selection yields}
 
\centering
\resizebox{1.0\linewidth}{!}{
\begin{tabular}{|l|c|c|c|c|c|c|c|}
\hline
 & \rotatebox{90}{DATA MET 2012A} & \rotatebox{90}{DATA MET 2012B} & \rotatebox{90}{DATA MET 2012C} & \rotatebox{90}{DATA MET 2012D} & \rotatebox{90}{DATA VBF-Parked 2012B} & \rotatebox{90}{DATA VBF-Parked 2012C} & \rotatebox{90}{DATA VBF-Parked 2012D} \\
\hline \hline
Vertex Filter & 3606391 & 15076553 & 21570165 & 59027309 & 132346320 & 228049748 & 308041846 \\
Event Quality Filters & 2658960 & 10926634 & 15555671 & 44411435 & 131554431 & 226680352 & 305918529 \\
ECAL Laser Filter & 2634271 & 10796003 & 15555671 & 44411435 & 131543040 & 226680352 & 305918529 \\
HCAL Laser Filter & 2634080 & 10796000 & 15554899 & 44411435 & 131543040 & 226679741 & 305918529 \\
L1T ETM & 2461217 & 9316076 & 13668424 & 37528140 & 88174347 & 160560859 & 227801622 \\
HLT Path & 97522 & 633305 & 1154795 & 2222706 & 75100422 & 137527238 & 152041761 \\
Veto(Electrons) & 96600 & 627254 & 1143298 & 2203960 & 74947192 & 137241812 & 151725585 \\
Veto(Muon) & 94864 & 619954 & 1129380 & 2187440 & 74913002 & 137179173 & 151652654 \\
Dijet & 28164 & 189270 & 358924 & 627128 & 23666926 & 43292391 & 42218637 \\
MET & 6252 & 32256 & 57239 & 71282 & 57929 & 102384 & 120600 \\
MET Significance & 3828 & 16475 & 29225 & 29964 & 24179 & 42683 & 41620 \\
$Min(\Delta\phi(Met,Jets))$ & 405 & 1602 & 3053 & 3047 & 1824 & 3452 & 3374 \\
\hline
\end{tabular}



}

\end{block}

\end{frame}



% ###################################################
\begin{frame}{Before typo fixing}
 
\begin{block}{Pre-selection yields}

\centering
% \resizebox{1.0\linewidth}{!}{
\begin{tabular}{|l|c|c|c|}
\hline
Dataset & Main Analysis & Cross Check Analysis & $1-\frac{CC}{Main}$ \\ 
\hline \hline
Prompt A &  405 &  406 & +0.247\% \\
Parked B & 1824 & 1836 & +0.658\% \\
Parked C & 3453 & 3466 & +0.376\% \\
Parked D & 3374 & 3392 & +0.533\% \\
\hline \hline
Total & 9056 & 9100 & +0.458\% \\
\hline
\end{tabular}
% }

\end{block}

\end{frame}

% ###################################################
\begin{frame}{After typo fixing}
 
\begin{block}{Pre-selection yields}

\centering
% \resizebox{1.0\linewidth}{!}{
\begin{tabular}{|l|c|c|c|}
\hline
Dataset & Main Analysis & Cross Check Analysis & $1-\frac{CC}{Main}$ \\ 
\hline \hline
Prompt A &  405 &  405 &  0.00\% \\
Parked B & 1824 & 1824 &  0.00\% \\
Parked C & 3453 & 3452 & -0.03\% \\
Parked D & 3374 & 3374 &  0.00\% \\
\hline \hline
Total & 9056 & 9055 & -0.01\% \\
\hline
\end{tabular}
% }

\end{block}

\end{frame}

% ###################################################
\begin{frame}{First at Electron+MET region} 
 
\begin{block}{Event selection}

The electron veto selection consists of the following cuts
\begin{itemize}
  \item Vertex cut 
  \item Event quality filters (MET Filters)
  \item ECAL and HCAL Laser filters
  \item L1T ETM $>=40$
  \item HLT path filter
  \item At least Electron Veto
  \item At least Electron Tight
  \item Muon Veto
  \item Dijet cut
  \begin{itemize}
      \item Lead dijet $ p_{\perp} > 50$ GeV
      \item Sub-lead dijet $ p_{\perp} > 45$ GeV
      \item Jets $ |\eta| < 4.7 $
      \item Dijet $ \Delta\eta < 3.6 $
      \item Dijet $ m_{jj} > 1200 $ GeV
  \end{itemize}
  \item MET $> 90$ GeV
  \item MET Significance $> 4.0$
  \item $ Min(\Delta\phi(MET,Jet_{p_{\perp}>30})))>2.3 $
\end{itemize}

\end{block}

\end{frame}

% ###################################################
\begin{frame}{First at Electron+MET region} 
 
\begin{block}{Yield}

\centering
\resizebox{1.0\linewidth}{!}{
\begin{tabular}{|l|c|c|c|c|c|c|c|}
\hline
 & \rotatebox{90}{DATA MET 2012A} & \rotatebox{90}{DATA MET 2012B} & \rotatebox{90}{DATA MET 2012C} & \rotatebox{90}{DATA MET 2012D} & \rotatebox{90}{DATA VBF-Parked 2012B} & \rotatebox{90}{DATA VBF-Parked 2012C} & \rotatebox{90}{DATA VBF-Parked 2012D} \\
\hline \hline
Vertex Filter & 3606391 & 15076553 & 21570165 & 59027309 & 132346320 & 228049748 & 308041846 \\
Event Quality Filters & 2658960 & 10926634 & 15555671 & 44411435 & 131554431 & 226680352 & 305918529 \\
ECAL Laser Filter & 2634271 & 10796003 & 15555671 & 44411435 & 131543040 & 226680352 & 305918529 \\
HCAL Laser Filter & 2634080 & 10796000 & 15554899 & 44411435 & 131543040 & 226679741 & 305918529 \\
L1T ETM& 2461217 & 9316076 & 13668424 & 37528140 & 88174347 & 160560859 & 227801622 \\
HLT Path & 97522 & 633305 & 1154795 & 2222706 & 75100422 & 137527238 & 152041761 \\
Select Electrons(Veto) & 922 & 6051 & 11497 & 18746 & 153230 & 285426 & 316176 \\
Select Electrons(Tight) & 453 & 2813 & 5280 & 6939 & 25375 & 47301 & 53678 \\
Veto(Muon) & 417 & 2692 & 5017 & 6670 & 25150 & 46895 & 53249 \\
MET & 287 & 1821 & 3395 & 3936 & 2243 & 4196 & 4837 \\
MET Significance & 133 & 843 & 1621 & 1681 & 863 & 1680 & 1727 \\
Dijet & 32 & 123 & 241 & 197 & 124 & 246 & 200 \\
$Min(\Delta\phi(Met,Jets))$ & 4 & 17 & 24 & 24 & 17 & 24 & 24 \\
\hline
\end{tabular}



}

\end{block}

\end{frame}

% ###################################################
\begin{frame}{Comparing to main analysis} 
 
\begin{block}{Yield}

\centering

\begin{tabular}{|l|c|c|c|}
\hline
Dataset & Main Analysis & Cross Check Analysis & $1-\frac{CC}{Main}$ \\ 
\hline \hline
Prompt A &  4 &  4 & 0.00\% \\
Parked B & 17 & 16 & 5.88\% \\
Parked C & 24 & 24 & 0.00\% \\
Parked D & 24 & 24 & 0.00\% \\
\hline \hline
Total & 69 & 68 & -1.45\% \\
\hline
\end{tabular}

\end{block}

\end{frame}

% ###################################################
\begin{frame}{Summary}
 
\begin{block}{Summary:}
 
\begin{itemize}
  \item Now better sync at pre-selection level
  \item Great sync at Electron-MET region
  \item Working on Muon+MET region
\end{itemize}

\end{block}

All results will be posted later today as usual at:
\begin{center}
https://twiki.cern.ch/twiki/bin/viewauth/CMS/VBFHInvParkedDataCrossCheck
\end{center}

\end{frame}

\end{document}
