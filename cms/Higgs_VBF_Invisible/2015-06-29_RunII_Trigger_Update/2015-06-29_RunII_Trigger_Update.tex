\documentclass[8pt]{beamer}
\usepackage[utf8]{inputenc}
\usepackage{xcolor}
\usepackage{colortbl}
\usepackage{epsfig}
% \usepackage{cancel}
\usepackage{ulem}
% \usepackage{threeparttable} % Joao Pela: 
\usepackage{amsmath}
\usepackage{hyperref}
\usepackage{siunitx}  % Allows easy x10^ for numbers
\usepackage{appendixnumberbeamer}

\usetheme{Madrid}

\author[J. Pela]{João Pela}
\title{Run II - Trigger Update}
\institute[ICL]{Imperial College London}
\date{2015-06-29}

% The log drawn in the upper right corner.
\logo{\includegraphics[height=0.115\paperheight]{img/Logo_CMSICL.png}}

\begin{document}
\setlength{\unitlength}{1mm}

% ###################################################
\begin{frame}
  \titlepage
\end{frame}

% ###################################################
\begin{frame}{Introduction}

\begin{block}
  
\begin{itemize}
  \item We were contacted by Pascal Vanlaer and David Sperka about a rate mismatch at 7E33 with PU20 and bunch separation 25 ns.
  \item They quoted the following ``official'' numbers for the rate:
  \begin{itemize}
    \item HLT\_DiPFJet40\_DEta3p5\_MJJ600\_PFMETNoMu80\_v1: $4.77 \pm 0.17$ versus our predicted 0.1
    \item HLT\_DiPFJet40\_DEta3p5\_MJJ600\_PFMETNoMu140\_v1: $0.26 \pm 0.01$ versus our predicted 4.5
  \end{itemize}
  \item Our prediction were NOT made for this instantaneous luminosity and PU scenario.
  \item I was asked to reproduce this rates and re-optimised the thresholds and prescales.
\end{itemize}

\end{block}

\end{frame}

% ###################################################
\begin{frame}{Software used}
  
With the help of Pascal, David and Jim I eventually got the CMSSW files produced with the latest HLT with characteristics: 
  
\begin{block}
  
\begin{itemize}
  \item CMSSW\_7\_4\_5
  \item HLT\_dev\_CMSSW\_7\_4\_0\_GRun\_V79
  \item HCAL Methods 0 and 3.
\end{itemize}
 
  
\end{block}

For now just using QCD (which is dominant) for rate calculation.

\end{frame}

% ###################################################
\begin{frame}{Example: HCAL Method 0 - HLT\_PFMET170\_NoiseCleaned\_v2}

\resizebox{\linewidth}{!}{
\begin{tabular}{|l||c|c||c|c|c|c|c|c|}
\hline
 & & & \multicolumn{6}{c|}{HLT\_PFMET170\_NoiseCleaned\_v2} \\
\hline
Sample & X Sec [pb] & Ev. Processed & Prescale & Ev Pass & Efficiency & Rate [Hz] & Rate Error [Hz] & Rate Error [\%] \\
\hline
\hline
VBF\_HToInv\_M-125  & 3.727 & 482896 & 1 & 43422 & 0.0899200 & 0.0023459 & 0.0000113 & 0.4799 \\
QCD\_Pt-50to80      & 22110000 & 3366038 & 1 & 0 & 0.0000000 & 0.0000000 & 0.0000000 & - \\
QCD\_Pt-80to120     & 3000114.3 & 4777532 & 1 & 2 & 0.0000004 & 0.0087915 & 0.0062165 & 70.7107 \\
QCD\_Pt-120to170    & 493200 & 4821602 & 1 & 76 & 0.0000158 & 0.0544181 & 0.0062422 & 11.4708 \\
QCD\_Pt-170to300    & 120300 & 1423446 & 1 & 320 & 0.0002248 & 0.1893096 & 0.0105827 & 5.5902 \\
QCD\_Pt-300to470    & 7475 & 2283385 & 1 & 4208 & 0.0018429 & 0.0964286 & 0.0014865 & 1.5416 \\
QCD\_Pt-470to600    & 587.1 & 2925749 & 1 & 17998 & 0.0061516 & 0.0252812 & 0.0001884 & 0.7454 \\
QCD\_Pt-600to800    & 167 & 2796098 & 1 & 34271 & 0.0122567 & 0.0143281 & 0.0000774 & 0.5402 \\
QCD\_Pt-800to1000   & 28.25 & 499998 & 1 & 13607 & 0.0272141 & 0.0053816 & 0.0000461 & 0.8573 \\
QCD\_Pt-1000to1400  & 8.195 & 499986 & 1 & 20919 & 0.0418392 & 0.0024001 & 0.0000166 & 0.6914 \\
QCD\_Pt-1400to1800  & 0.7346 & 499994 & 1 & 41431 & 0.0828630 & 0.0004261 & 0.0000021 & 0.4913 \\
QCD\_Pt-1800        & 0.1091 & 488939 & 1 & 67211 & 0.1374630 & 0.0001050 & 0.0000004 & 0.3857 \\
\hline
\hline
Total Rate &  &  &  &  &  & 0.3968698 & 0.0248703 & 6.27\% \\
\hline
\end{tabular}
}


\end{frame}

% ###################################################
\begin{frame}{Conclusions}
 
\begin{itemize}
  \item All the rates I obtained are for *\_v2 of this paths while from TSG they are all *\_v1 I am unsure of what changed.
  \item The rates obtained for HLT\_PFMET170\_NoiseCleaned\_v2 
  \begin{itemize}
    \item HCAL Method 0: 0.397+/-0.025 Hz 
    \item HCAL Method 3: 0.261+/-0.020 Hz.
    \item From TSG: 0.37+/-0.02 Hz, which matches well for HCAL Method 0.
  \end{itemize}
  \item The rate obtained for HLT\_DiPFJet40\_DEta3p5\_MJJ600\_PFMETNoMu140\_v2
  \begin{itemize}
    \item HCAL Method 0: 0.441+/-0.035 Hz 
    \item HCAL Method 3: 0.242+/-0.024 Hz
    \item From TSG: 0.26+/-0.01 Hz, which matches well for HCAL Method 3.
  \end{itemize}
  \item The rate obtained for HLT\_DiPFJet40\_DEta3p5\_MJJ600\_PFMETNoMu80\_v2
  \begin{itemize}
    \item HCAL Method 0: 13.355+/-0.370 Hz (assuming HLT prescale 3 like in like TSG)
    \item HCAL Method 3: 5.150+/-0.200 Hz (assuming HLT prescale 3 like TSG)
    \item From  TSG is 4.77+/-0.17 Hz, which matches well for HCAL Method 3.
  \end{itemize}
  \item Signal efficiency (mH=125 GeV) for HLT\_PFMET170\_NoiseCleaned\_v2 + HLT\_DiPFJet40\_DEta3p5\_MJJ600\_PFMETNoMu140\_v2 is:
  \begin{itemize}
    \item HCAL Method 0: 10.60\% here the dedicate HLT path added an additional 17.85\% of signal
    \item HCAL Method 3: 9.40\% here the dedicate HLT path added an additional 17.96\% of signal
  \end{itemize}
  \item HCAL Method 3 reduces rate versus HCAL Method 0
  \begin{itemize}
    \item HLT\_PFMET170\_NoiseCleaned\_v2: -34\%
    \item HLT\_DiPFJet40\_DEta3p5\_MJJ600\_PFMETNoMu140\_v2: -45\%
    \item HLT\_DiPFJet40\_DEta3p5\_MJJ600\_PFMETNoMu80\_v2: -61\%
  \end{itemize}
\end{itemize}

\end{frame}

% ###################################################
\begin{frame}{Summary}
 
\begin{block}{Summary:}
 
\begin{itemize}
  \item Have not the capability to obtain rates and efficiencies with the latest HLT code
  \item Efficiencies look reasonable but can be improved. 
\end{itemize}

\end{block}

\begin{block}{Next steps:}
 
\begin{itemize}
  \item Was asked to look at more recent samples with better QCD modelling and re-optimize. 
\end{itemize}

\end{block}

\end{frame}

% ###################################################
\appendix
% ###################################################
\begin{frame}
 
\begin{block}

\begin{center}Backup Slides\end{center}

\end{block}

\end{frame}

% ###################################################
\begin{frame}{HCAL Method 0 - HLT\_PFMET170\_NoiseCleaned\_v2}

\resizebox{\linewidth}{!}{
\begin{tabular}{|l||c|c||c|c|c|c|c|c|}
\hline
 & & & \multicolumn{6}{c|}{HLT\_PFMET170\_NoiseCleaned\_v2} \\
\hline
Sample & X Sec [pb] & Ev. Processed & Prescale & Ev Pass & Efficiency & Rate [Hz] & Rate Error [Hz] & Rate Error [\%] \\
\hline
\hline
VBF\_HToInv\_M-125  & 3.727 & 482896 & 1 & 43422 & 0.0899200 & 0.0023459 & 0.0000113 & 0.4799 \\
QCD\_Pt-50to80      & 22110000 & 3366038 & 1 & 0 & 0.0000000 & 0.0000000 & 0.0000000 & - \\
QCD\_Pt-80to120     & 3000114.3 & 4777532 & 1 & 2 & 0.0000004 & 0.0087915 & 0.0062165 & 70.7107 \\
QCD\_Pt-120to170    & 493200 & 4821602 & 1 & 76 & 0.0000158 & 0.0544181 & 0.0062422 & 11.4708 \\
QCD\_Pt-170to300    & 120300 & 1423446 & 1 & 320 & 0.0002248 & 0.1893096 & 0.0105827 & 5.5902 \\
QCD\_Pt-300to470    & 7475 & 2283385 & 1 & 4208 & 0.0018429 & 0.0964286 & 0.0014865 & 1.5416 \\
QCD\_Pt-470to600    & 587.1 & 2925749 & 1 & 17998 & 0.0061516 & 0.0252812 & 0.0001884 & 0.7454 \\
QCD\_Pt-600to800    & 167 & 2796098 & 1 & 34271 & 0.0122567 & 0.0143281 & 0.0000774 & 0.5402 \\
QCD\_Pt-800to1000   & 28.25 & 499998 & 1 & 13607 & 0.0272141 & 0.0053816 & 0.0000461 & 0.8573 \\
QCD\_Pt-1000to1400  & 8.195 & 499986 & 1 & 20919 & 0.0418392 & 0.0024001 & 0.0000166 & 0.6914 \\
QCD\_Pt-1400to1800  & 0.7346 & 499994 & 1 & 41431 & 0.0828630 & 0.0004261 & 0.0000021 & 0.4913 \\
QCD\_Pt-1800        & 0.1091 & 488939 & 1 & 67211 & 0.1374630 & 0.0001050 & 0.0000004 & 0.3857 \\
\hline
\hline
Total Rate &  &  &  &  &  & 0.3968698 & 0.0248703 & 6.27\% \\
\hline
\end{tabular}
}


\end{frame}

\begin{frame}{HCAL Method 0 - HLT\_DiPFJet40\_DEta3p5\_MJJ600\_PFMETNoMu140\_v2}

\resizebox{\linewidth}{!}{
\begin{tabular}{|l||c|c||c|c|c|c|c|c|}
\hline
 & & & \multicolumn{6}{c|}{HLT\_DiPFJet40\_DEta3p5\_MJJ600\_PFMETNoMu140\_v2} \\
\hline
Sample & X Sec [pb] & Ev. Processed & Prescale & Ev Pass & Efficiency & Rate [Hz] & Rate Error [Hz] & Rate Error [\%] \\
\hline
\hline
VBF\_HToInv\_M-125  & 3.727 & 482896 & 1 & 24112 & 0.0499321 & 0.0013027 & 0.0000084 & 0.6440 \\
QCD\_Pt-50to80      & 22110000 & 3366038 & 1 & 0 & 0.0000000 & 0.0000000 & 0.0000000 & - \\
QCD\_Pt-80to120     & 3000114.3 & 4777532 & 1 & 10 & 0.0000021 & 0.0439574 & 0.0139006 & 31.6228 \\
QCD\_Pt-120to170    & 493200 & 4821602 & 1 & 140 & 0.0000290 & 0.1002439 & 0.0084722 & 8.4515 \\
QCD\_Pt-170to300    & 120300 & 1423446 & 1 & 362 & 0.0002543 & 0.2141565 & 0.0112558 & 5.2559 \\
QCD\_Pt-300to470    & 7475 & 2283385 & 1 & 2929 & 0.0012827 & 0.0671196 & 0.0012402 & 1.8477 \\
QCD\_Pt-470to600    & 587.1 & 2925749 & 1 & 6696 & 0.0022886 & 0.0094056 & 0.0001149 & 1.2221 \\
QCD\_Pt-600to800    & 167 & 2796098 & 1 & 10232 & 0.0036594 & 0.0042778 & 0.0000423 & 0.9886 \\
QCD\_Pt-800to1000   & 28.25 & 499998 & 1 & 3001 & 0.0060020 & 0.0011869 & 0.0000217 & 1.8254 \\
QCD\_Pt-1000to1400  & 8.195 & 499986 & 1 & 4502 & 0.0090043 & 0.0005165 & 0.0000077 & 1.4904 \\
QCD\_Pt-1400to1800  & 0.7346 & 499994 & 1 & 6354 & 0.0127082 & 0.0000653 & 0.0000008 & 1.2545 \\
QCD\_Pt-1800        & 0.1091 & 488939 & 1 & 6993 & 0.0143024 & 0.0000109 & 0.0000001 & 1.1958 \\
\hline
\hline
Total Rate &  &  &  &  &  & 0.4409406 & 0.0350647 & 7.95\% \\
\hline
\end{tabular}
}


\end{frame}

\begin{frame}{HCAL Method 0 - HLT\_DiPFJet40\_DEta3p5\_MJJ600\_PFMETNoMu80\_v2}

\resizebox{\linewidth}{!}{
\begin{tabular}{|l||c|c||c|c|c|c|c|c|}
\hline
 & & & \multicolumn{6}{c|}{HLT\_DiPFJet40\_DEta3p5\_MJJ600\_PFMETNoMu80\_v2} \\
\hline
Sample & X Sec [pb] & Ev. Processed & Prescale & Ev Pass & Efficiency & Rate [Hz] & Rate Error [Hz] & Rate Error [\%] \\
\hline
\hline
VBF\_HToInv\_M-125  & 3.727 & 482896 & 3 & 43410 & 0.0898951 & 0.0007818 & 0.0000038 & 0.4800 \\
QCD\_Pt-50to80      & 22110000 & 3366038 & 3 & 248 & 0.0000737 & 3.8010028 & 0.2413639 & 6.3500 \\
QCD\_Pt-80to120     & 3000114.3 & 4777532 & 3 & 3038 & 0.0006359 & 4.4514218 & 0.0807616 & 1.8143 \\
QCD\_Pt-120to170    & 493200 & 4821602 & 3 & 13007 & 0.0026977 & 3.1044569 & 0.0272206 & 0.8768 \\
QCD\_Pt-170to300    & 120300 & 1423446 & 3 & 9107 & 0.0063979 & 1.7958777 & 0.0188187 & 1.0479 \\
QCD\_Pt-300to470    & 7475 & 2283385 & 3 & 22836 & 0.0100009 & 0.1744331 & 0.0011543 & 0.6617 \\
QCD\_Pt-470to600    & 587.1 & 2925749 & 3 & 39642 & 0.0135494 & 0.0185613 & 0.0000932 & 0.5023 \\
QCD\_Pt-600to800    & 167 & 2796098 & 3 & 50254 & 0.0179729 & 0.0070034 & 0.0000312 & 0.4461 \\
QCD\_Pt-800to1000   & 28.25 & 499998 & 3 & 11561 & 0.0231221 & 0.0015241 & 0.0000142 & 0.9300 \\
QCD\_Pt-1000to1400  & 8.195 & 499986 & 3 & 14678 & 0.0293568 & 0.0005614 & 0.0000046 & 0.8254 \\
QCD\_Pt-1400to1800  & 0.7346 & 499994 & 3 & 15996 & 0.0319924 & 0.0000548 & 0.0000004 & 0.7907 \\
QCD\_Pt-1800        & 0.1091 & 488939 & 3 & 14455 & 0.0295640 & 0.0000075 & 0.0000001 & 0.8317 \\
\hline
\hline
Total Rate &  &  &  &  &  & 13.3549049 & 0.3694666 & 2.77\% \\
\hline
\end{tabular}
}


\end{frame}

\begin{frame}{HCAL Method 3 - HLT\_PFMET170\_NoiseCleaned\_v2}

\resizebox{\linewidth}{!}{
\begin{tabular}{|l||c|c||c|c|c|c|c|c|}
\hline
 & & & \multicolumn{6}{c|}{HLT\_PFMET170\_NoiseCleaned\_v2} \\
\hline
Sample & X Sec [pb] & Ev. Processed & Prescale & Ev Pass & Efficiency & Rate [Hz] & Rate Error [Hz] & Rate Error [\%] \\
\hline
\hline
VBF\_HToInv\_M-125  & 3.727 & 482896 & 1 & 38491 & 0.0797087 & 0.0020795 & 0.0000106 & 0.5097 \\
QCD\_Pt-50to80      & 22110000 & 3484212 & 1 & 0 & 0.0000000 & 0.0000000 & 0.0000000 & - \\
QCD\_Pt-80to120     & 3000114.3 & 4777411 & 1 & 2 & 0.0000004 & 0.0087917 & 0.0062167 & 70.7107 \\
QCD\_Pt-120to170    & 493200 & 4821602 & 1 & 31 & 0.0000064 & 0.0221969 & 0.0039867 & 17.9605 \\
QCD\_Pt-170to300    & 120300 & 1423446 & 1 & 194 & 0.0001363 & 0.1147689 & 0.0082399 & 7.1796 \\
QCD\_Pt-300to470    & 7475 & 2247929 & 1 & 3292 & 0.0014645 & 0.0766278 & 0.0013355 & 1.7429 \\
QCD\_Pt-470to600    & 587.1 & 2925749 & 1 & 16705 & 0.0057096 & 0.0234649 & 0.0001816 & 0.7737 \\
QCD\_Pt-600to800    & 167 & 2925749 & 1 & 16705 & 0.0057096 & 0.0066746 & 0.0000516 & 0.7737 \\
QCD\_Pt-800to1000   & 28.25 & 499998 & 1 & 13607 & 0.0272141 & 0.0053816 & 0.0000461 & 0.8573 \\
QCD\_Pt-1000to1400  & 8.195 & 499986 & 1 & 28225 & 0.0564516 & 0.0032383 & 0.0000193 & 0.5952 \\
QCD\_Pt-1400to1800  & 0.7346 & 488314 & 1 & 58626 & 0.1200580 & 0.0006174 & 0.0000025 & 0.4130 \\
QCD\_Pt-1800        & 0.1091 & 488939 & 1 & 97091 & 0.1985749 & 0.0001517 & 0.0000005 & 0.3209 \\
\hline
\hline
Total Rate &  &  &  &  &  & 0.2619138 & 0.0200911 & 7.67\% \\
\hline
\end{tabular}
}


\end{frame}

\begin{frame}{HCAL Method3 - HLT\_DiPFJet40\_DEta3p5\_MJJ600\_PFMETNoMu140\_v2}

\resizebox{\linewidth}{!}{
\begin{tabular}{|l||c|c||c|c|c|c|c|c|}
\hline
 & & & \multicolumn{6}{c|}{HLT\_DiPFJet40\_DEta3p5\_MJJ600\_PFMETNoMu140\_v2} \\
\hline
Sample & X Sec [pb] & Ev. Processed & Prescale & Ev Pass & Efficiency & Rate [Hz] & Rate Error [Hz] & Rate Error [\%] \\
\hline
\hline
VBF\_HToInv\_M-125  & 3.727 & 482896 & 1 & 20799 & 0.0430714 & 0.0011237 & 0.0000078 & 0.6934 \\
QCD\_Pt-50to80      & 22110000 & 3484212 & 1 & 0 & 0.0000000 & 0.0000000 & 0.0000000 & - \\
QCD\_Pt-80to120     & 3000114.3 & 4777411 & 1 & 4 & 0.0000008 & 0.0175834 & 0.0087917 & 50.0000 \\
QCD\_Pt-120to170    & 493200 & 4821602 & 1 & 64 & 0.0000133 & 0.0458258 & 0.0057282 & 12.5000 \\
QCD\_Pt-170to300    & 120300 & 1423446 & 1 & 198 & 0.0001391 & 0.1171353 & 0.0083244 & 7.1067 \\
QCD\_Pt-300to470    & 7475 & 2247929 & 1 & 2103 & 0.0009355 & 0.0489515 & 0.0010674 & 2.1806 \\
QCD\_Pt-470to600    & 587.1 & 2925749 & 1 & 5755 & 0.0019670 & 0.0080839 & 0.0001066 & 1.3182 \\
QCD\_Pt-600to800    & 167 & 2925749 & 1 & 5755 & 0.0019670 & 0.0022994 & 0.0000303 & 1.3182 \\
QCD\_Pt-800to1000   & 28.25 & 499998 & 1 & 3001 & 0.0060020 & 0.0011869 & 0.0000217 & 1.8254 \\
QCD\_Pt-1000to1400  & 8.195 & 499986 & 1 & 4868 & 0.0097363 & 0.0005585 & 0.0000080 & 1.4333 \\
QCD\_Pt-1400to1800  & 0.7346 & 488314 & 1 & 7039 & 0.0144149 & 0.0000741 & 0.0000009 & 1.1919 \\
QCD\_Pt-1800        & 0.1091 & 488939 & 1 & 7929 & 0.0162167 & 0.0000124 & 0.0000001 & 1.1230 \\
\hline
\hline
Total Rate &  &  &  &  &  & 0.2417112 & 0.0240872 & 9.97\% \\
\hline
\end{tabular}
}


\end{frame}

\begin{frame}{HCAL Method 3 - HLT\_DiPFJet40\_DEta3p5\_MJJ600\_PFMETNoMu80\_v2}

\resizebox{\linewidth}{!}{
\begin{tabular}{|l||c|c||c|c|c|c|c|c|}
\hline
 & & & \multicolumn{6}{c|}{HLT\_DiPFJet40\_DEta3p5\_MJJ600\_PFMETNoMu80\_v2} \\
\hline
Sample & X Sec [pb] & Ev. Processed & Prescale & Ev Pass & Efficiency & Rate [Hz] & Rate Error [Hz] & Rate Error [\%] \\
\hline
\hline
VBF\_HToInv\_M-125  & 3.727 & 482896 & 3 & 38721 & 0.0801850 & 0.0006973 & 0.0000035 & 0.5082 \\
QCD\_Pt-50to80      & 22110000 & 3484212 & 3 & 64 & 0.0000184 & 0.9476346 & 0.1184543 & 12.5000 \\
QCD\_Pt-80to120     & 3000114.3 & 4777411 & 3 & 1045 & 0.0002187 & 1.5312224 & 0.0473675 & 3.0934 \\
QCD\_Pt-120to170    & 493200 & 4821602 & 3 & 5966 & 0.0012373 & 1.4239402 & 0.0184353 & 1.2947 \\
QCD\_Pt-170to300    & 120300 & 1423446 & 3 & 5523 & 0.0038800 & 1.0891218 & 0.0146551 & 1.3456 \\
QCD\_Pt-300to470    & 7475 & 2247929 & 3 & 17525 & 0.0077961 & 0.1359764 & 0.0010272 & 0.7554 \\
QCD\_Pt-470to600    & 587.1 & 2925749 & 3 & 34115 & 0.0116603 & 0.0159734 & 0.0000865 & 0.5414 \\
QCD\_Pt-600to800    & 167 & 2925749 & 3 & 34115 & 0.0116603 & 0.0045436 & 0.0000246 & 0.5414 \\
QCD\_Pt-800to1000   & 28.25 & 499998 & 3 & 11561 & 0.0231221 & 0.0015241 & 0.0000142 & 0.9300 \\
QCD\_Pt-1000to1400  & 8.195 & 499986 & 3 & 14383 & 0.0287668 & 0.0005501 & 0.0000046 & 0.8338 \\
QCD\_Pt-1400to1800  & 0.7346 & 488314 & 3 & 15332 & 0.0313978 & 0.0000538 & 0.0000004 & 0.8076 \\
QCD\_Pt-1800        & 0.1091 & 488939 & 3 & 14073 & 0.0287827 & 0.0000073 & 0.0000001 & 0.8430 \\
\hline
\hline
Total Rate &  &  &  &  &  & 5.1505478 & 0.2000732 & 3.88\% \\
\hline
\end{tabular}
}


\end{frame}

\end{document}
