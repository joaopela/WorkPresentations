\documentclass[8pt]{beamer}
\usepackage[utf8]{inputenc}
\usepackage{xcolor}
\usepackage{colortbl}
\usepackage{epsfig}
% \usepackage{cancel}
\usepackage{ulem}
% \usepackage{threeparttable} % Joao Pela: 
\usepackage{amsmath}
\usepackage{hyperref}
% \usepackage{feynmp}         % For latex produced Feynman Diagrams

% Rule for feynmp diagrams to be considered graphics
% \DeclareGraphicsRule{*}{mps}{*}{}
% 
% % New compile sequence for feynmp
% \makeatletter
% \def\endfmffile{%
%   \fmfcmd{\p@rcent\space the end.^^J%
%           end.^^J%
%           endinput;}%
%   \if@fmfio
%     \immediate\closeout\@outfmf
%   \fi
%   \ifnum\pdfshellescape=\@ne
%     \immediate\write18{mpost \thefmffile}%
%   \fi}
% \makeatother

\usetheme{Madrid}

\author[João Pela]{J. Pela}
\title[]{MC VBF+MET QCD Samples}
\institute{Imperial College London}
\date{2013-11-17}

% The log drawn in the upper right corner.
\logo{\includegraphics[height=0.115\paperheight]{img/Logo_CMSICL.png}}

\begin{document}
\setlength{\unitlength}{1mm}

% ###################################################
\begin{frame}
  \titlepage
\end{frame}

% ###################################################
\begin{frame}{Introduction and Motivation}

\begin{block}{Motivation}

  \begin{itemize}
    \item Create a set of QCD MC samples that would model adequately events passing our selection.
    \item Generate enough statistics to represent 2012 dataset (~20 $fb^{-1}$)
  \end{itemize}
  
\end{block}

\begin{block}{Caveats:}

  \begin{itemize}
    \item Huge cross section of QCD
    \item We cannot do post RECO selection since this would too time consuming.
    \item Need to define a QCD Hard scattering minimum to avoid rising cross section of low $p_T$
          interactions where VBF+MET type events are not likely anyway
  \end{itemize}
  
\end{block}

\end{frame}

% ###################################################
\begin{frame}{Methodology}

We will be looking at gen level particles only to avoid the RECO process

\begin{block}{MET}

  \begin{itemize}
    \item Select all produced neutrinos and add them vectorially.
    \item Determine their $p_T$.
  \end{itemize}

\end{block}

\begin{block}{VBF Jets}

  \begin{itemize}
    \item Run AK5 genJets (without neutrinos) over gen-particles.
    \item Select all jets with a given $p_T$ and $|\eta|$.
    \item Calculate $\Delta\eta$ and $M_{jj}$ for all possible dijet combinations.
    \item Accept event if one of combinations passes all requirements.
  \end{itemize}

\end{block}

\begin{block}{Caveats:}
 
  \begin{itemize}
    \item Thresholds must be set carefully and low enough to represent the QCD that actually passes the
          analysis (at some cut L1+HLT, dijet, etc).
    \item Trigger/variable turn on and efficiency should be taken into account.
  \end{itemize}

\end{block}

\end{frame}

% ###################################################
\begin{frame}{QCD Cross Sections and event predictions for 20 $fb^{-1}$}

From the current samples and cross sections we can easily extrapolate what would be the expected
number of events for each $p_T$ hat for an integrated luminosity of 20 $fb^{-1}$.

\begin{block}

\centering
\begin{tabular}{|l||c|c|}
\hline
\hline \hline
Sample & Cross Section (pb) & Events for 20 $fb^{-1}$ \\
\hline \hline
QCD-Pt-30to50-pythia6     & 66285328    & 1325706560000 \\
QCD-Pt-50to80-pythia6     & 8148778     & 162975560000 \\
QCD-Pt-80to120-pythia6    & 1033680     & 20673600000 \\
QCD-Pt-120to170-pythia6   & 156293,3    & 3125866000 \\
QCD-Pt-170to300-pythia6   & 34138,15    & 682763000 \\
QCD-Pt-300to470-pythia6   & 1759,549    & 35190980 \\
QCD-Pt-470to600-pythia6   & 113,8791    & 2277582 \\
QCD-Pt-600to800-pythia6   & 26,9921     & 539842 \\
QCD-Pt-800to1000-pythia6  & 3,550036    & 71000,72 \\
QCD-Pt-1000to1400-pythia6 & 0,737844    & 14756,88 \\
QCD-Pt-1400to1800-pythia6 & 0,03352235  & 670,45 \\
QCD-Pt-1800-pythia6       & 0,001829005 & 36,58 \\
\hline
\end{tabular}

\end{block}

If we consider a minimum $p_T$ for hard scattering of 80 GeV the total cross section for 1226016 $pb$ which
implies we need a a rejection factor of 10000 to be able to produce a 20 $fb^{-1}$ sample with 2.5M 
events.

\end{frame}

% ###################################################
\begin{frame}{First tests}
 
table and discussion of choice of values
 
\end{frame}

% ###################################################
\begin{frame}{Filter Efficiency per $p_T$ hat}
 
\begin{tabular}{|l||c|c|c|c|c|c|c|c|}
\hline
\multicolumn{9}{|c|}{Cross Sections} \\
\hline \hline
                 Sample & Gen. Ev & Pass MET & Pass Dijet &   Factor & Sample & Ev. Size &  Sample & Sample size \\
\hline \hline
QCD-Pt-50to80-pythia6   & 1000000 &      127 &          3 & 0,000003 & 488927 &    82,00 & 1000000 &  82,00 \\
QCD-Pt-80to120-pythia6  & 1000000 &     1172 &         41 & 0,000041 & 847618 &    70,73 & 2000000 & 141,46 \\
QCD-Pt-120to170-pythia6 & 1000000 &     4276 &        293 & 0,000293 & 915879 &    68,26 & 2000000 & 136,52 \\
QCD-Pt-170to300-pythia6 & 1000000 &     9315 &       1012 & 0,001012 & 690956 &    66,21 & 3500000 & 231,72 \\
QCD-Pt-300to470-pythia6 & 1000000 &    17956 &       2598 & 0,002598 &  91426 &    67,36 &  200000 &  13,47 \\
QCD-Pt-470to600-pythia6 & 1000000 &    23913 &       4187 & 0,004187 &   9536 &    68,55 &  100000 &   6,85 \\
\hline
\end{tabular}
 
\end{frame}

% ###################################################
\begin{frame}{Steps for production}
 
Explain steps to be done and CMSSW versions
 
\end{frame}

% ###################################################
\begin{frame}{Production Status}
 
Explain steps to be done and CMSSW versions
 
\end{frame}

% ###################################################
\begin{frame}{Problems found}
 
PU etc.
 
\end{frame}

% ###################################################
\begin{frame}{Conclusions}

\begin{block}

\begin{itemize}
  \item 
\end{itemize} 

\end{block}

\end{frame}

\end{document}