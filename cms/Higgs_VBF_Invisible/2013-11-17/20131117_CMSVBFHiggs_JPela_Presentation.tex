\documentclass[8pt]{beamer}
\usepackage[utf8]{inputenc}
\usepackage{xcolor}
\usepackage{colortbl}
\usepackage{epsfig}
% \usepackage{cancel}
\usepackage{ulem}
% \usepackage{threeparttable} % Joao Pela: 
\usepackage{amsmath}
\usepackage{hyperref}
% \usepackage{feynmp}         % For latex produced Feynman Diagrams

% Rule for feynmp diagrams to be considered graphics
% \DeclareGraphicsRule{*}{mps}{*}{}
% 
% % New compile sequence for feynmp
% \makeatletter
% \def\endfmffile{%
%   \fmfcmd{\p@rcent\space the end.^^J%
%           end.^^J%
%           endinput;}%
%   \if@fmfio
%     \immediate\closeout\@outfmf
%   \fi
%   \ifnum\pdfshellescape=\@ne
%     \immediate\write18{mpost \thefmffile}%
%   \fi}
% \makeatother

\usetheme{Madrid}

\author[João Pela]{J. Pela}
\title[]{MC VBF+MET QCD Samples}
\institute{Imperial College London}
\date{2013-11-18}

% The log drawn in the upper right corner.
\logo{\includegraphics[height=0.115\paperheight]{img/Logo_CMSICL.png}}

\begin{document}
\setlength{\unitlength}{1mm}

% ###################################################
\begin{frame}
  \titlepage
\end{frame}

% ###################################################
\begin{frame}{Introduction and Motivation}

\begin{block}{Motivation}

  \begin{itemize}
    \item Create a set of QCD MC samples that would model adequately events passing our selection.
    \item Generate enough statistics to represent 2012 dataset (~20 $fb^{-1}$)
  \end{itemize}
  
\end{block}

\begin{block}{Caveats:}

  \begin{itemize}
    \item Huge cross section of QCD
    \item We cannot do post RECO selection since this would too time consuming.
    \item Need to define a QCD Hard scattering minimum to avoid rising cross section of low $p_T$
          interactions where VBF+MET type events are not likely anyway
  \end{itemize}
  
\end{block}

\end{frame}

% ###################################################
\begin{frame}{Methodology}

We will be looking at gen level particles only to avoid the RECO process

\begin{block}{MET}

  \begin{itemize}
    \item Select all produced neutrinos and add them vectorially.
    \item Determine their $p_T$.
  \end{itemize}

\end{block}

\begin{block}{VBF Jets}

  \begin{itemize}
    \item Run AK5 genJets (without neutrinos) over gen-particles.
    \item Select all jets with a given $p_T$ and $|\eta|$.
    \item Calculate $\Delta\eta$ and $M_{jj}$ for all possible dijet combinations.
    \item Accept event if one of combinations passes all requirements.
  \end{itemize}

\end{block}

\begin{block}{Caveats:}
 
  \begin{itemize}
    \item Thresholds must be set carefully and low enough to represent the QCD that actually passes the
          analysis (at some cut L1+HLT, dijet, etc).
    \item Trigger/variable turn on and efficiency should be taken into account.
  \end{itemize}

\end{block}

\end{frame}

% ###################################################
\begin{frame}{QCD Cross Sections and event predictions for 20 $fb^{-1}$}

From the current samples and cross sections we can easily extrapolate what would be the expected
number of events for each $p_T$ hat for an integrated luminosity of 20 $fb^{-1}$.

\begin{block}

\centering
\begin{tabular}{|l||c|c|}
\hline
\hline \hline
Sample & Cross Section (pb) & Events for 20 $fb^{-1}$ \\
\hline \hline
QCD-Pt-30to50-pythia6     & 66285328    & 1325706560000 \\
QCD-Pt-50to80-pythia6     & 8148778     & 162975560000 \\
QCD-Pt-80to120-pythia6    & 1033680     & 20673600000 \\
QCD-Pt-120to170-pythia6   & 156293,3    & 3125866000 \\
QCD-Pt-170to300-pythia6   & 34138,15    & 682763000 \\
QCD-Pt-300to470-pythia6   & 1759,549    & 35190980 \\
QCD-Pt-470to600-pythia6   & 113,8791    & 2277582 \\
QCD-Pt-600to800-pythia6   & 26,9921     & 539842 \\
QCD-Pt-800to1000-pythia6  & 3,550036    & 71000,72 \\
QCD-Pt-1000to1400-pythia6 & 0,737844    & 14756,88 \\
QCD-Pt-1400to1800-pythia6 & 0,03352235  & 670,45 \\
QCD-Pt-1800-pythia6       & 0,001829005 & 36,58 \\
\hline
\end{tabular}

\end{block}

If we consider a minimum $p_T$ for hard scattering of 80 GeV the total cross section for 1226016 $pb$ which
implies we need a a rejection factor of 10000 to be able to produce a 20 $fb^{-1}$ sample with 2.5M 
events.

\end{frame}

% ###################################################
\begin{frame}{Filter Efficiency per $p_T$ hat}
  
I tested a working point similar to trigger thresholds:
\begin{block}{Filter conditions:}

  \begin{itemize}
    \item $MET(neutrinos)>40$ $GeV$
    \item Jets $p_T>20$ $GeV$ and $|\eta|<5.0$
    \item $\Delta\eta>3.2$ and $M_{jj}>700$ $GeV$
  \end{itemize}
 
\end{block}

\begin{block}{Efficiency:}
 
\resizebox{\linewidth}{!}{
\begin{tabular}{|l||c|c|c|c|c|}
\hline
\multicolumn{6}{|c|}{Cross Sections} \\
\hline \hline
                 Sample & Gen. Ev & Pass MET & Pass Dijet &   Factor & Sample \\
\hline \hline
QCD-Pt-50to80-pythia6   & 1000000 &      127 &          3 & 0,000003 & 488927 \\
QCD-Pt-80to120-pythia6  & 1000000 &     1172 &         41 & 0,000041 & 847618 \\
QCD-Pt-120to170-pythia6 & 1000000 &     4276 &        293 & 0,000293 & 915879 \\
QCD-Pt-170to300-pythia6 & 1000000 &     9315 &       1012 & 0,001012 & 690956 \\
QCD-Pt-300to470-pythia6 & 1000000 &    17956 &       2598 & 0,002598 &  91426 \\
QCD-Pt-470to600-pythia6 & 1000000 &    23913 &       4187 & 0,004187 &   9536 \\
\hline
\end{tabular}
} 
\end{block}

With the obtained rejection factor we can generate the equivalent of 20 $fb^{-1}$ with 3M events I will have to generate at least twice that.

\end{frame}

% ###################################################
\begin{frame}{Steps for production}
 
I am replicating the production process of Summer 2012 QCD samples Production so they match the currently used QCD samples. Samples are produced over 3 steps:

\begin{block}{Step 1 - Hard process}
 
  \begin{itemize}
    \item Made with CMSSW\_5\_0\_0\_patch2.
    \item Using Pythia6 QCD normal configuration fragments with 2 additional filters over GEN level.
  \end{itemize} 
 
\end{block}

\begin{block}{Step 2 - Pileup addition}
 
  \begin{itemize}
    \item Made with CMSSW\_5\_3\_2\_patch4
    \item Doing REDIGI using frontier tags START53\_V7A and for pileup 2012\_Summer\_50ns\_PoissonOOTPU
  \end{itemize} 
 
\end{block}

\begin{block}{Step 1 - RECO}
 
  \begin{itemize}
    \item Made with CMSSW\_5\_3\_2\_patch4 
    \item RECO process and output of AODSIM output
  \end{itemize} 
 
\end{block}
 
\end{frame}

% ###################################################
\begin{frame}{Production Status}
 
\begin{block}{Step I - Status}

\centering
\begin{tabular}{|l||c|c|c|c|c|}
\hline
Sample          & Jobs & Done &  Events & Target & Int. Lumi ($fb^{-1}$) \\
\hline \hline
QCD-Pt-80to120  & 5000 &  206 &       ? & 847618 &  \\
QCD-Pt-120to170 & 3500 & 3500 & 1985505 & 915879 &  43.36 \\
QCD-Pt-170to300 & 5000 & 5000 & 3465175 & 690956 & 100.30 \\
QCD-Pt-300to470 &  400 &  400 &  214621 &  91426 &  46.95 \\
QCD-Pt-470to600 &  250 &  250 &  104264 &   9536 & 218.67 \\
\hline
\end{tabular}

\end{block}

All samples except QCD-Pt-80to120 (had to be resubmitted) have finished step 1 and have been published on cms\_dbs\_ph\_analysis\_01.

\end{frame}

% ###################################################
\begin{frame}{Conclusions}

\begin{block}{Status and next steps}

\begin{itemize}
  \item Working point with the necessary rejection factor was determined and samples are now under production
  \item Currently stuck at step 1 since PU samples are needed for the mixing (in contact with experts)
  \item Validation in start in parallel to production.
  \item To note the this samples will simulated events with real MET and real genJets, and not events where full MET or jets are faked.
\end{itemize} 

\end{block}

\end{frame}

\end{document}